%%=============================================================================
%% Voorwoord
%%=============================================================================

\chapter*{\IfLanguageName{dutch}{Woord vooraf}{Preface}}
\label{ch:voorwoord}

%% TODO:
%% Het voorwoord is het enige deel van de bachelorproef waar je vanuit je
%% eigen standpunt (``ik-vorm'') mag schrijven. Je kan hier bv. motiveren
%% waarom jij het onderwerp wil bespreken.
%% Vergeet ook niet te bedanken wie je geholpen/gesteund/... heeft

Deze scriptie is geschreven ter afsluiting van de opleiding Toegepaste Informatica van de Hogeschool Gent.

Het onderwerp van deze scriptie heb ik bedacht omdat ik al interesse had in het werken met IDEs en dit op een zo efficiënt mogelijke manier te doen. Op het eerste zicht was er niet veel informatie over dit onderwerp terug te vinden, zeker niet over uitvoeren van de performantie testen, maar ik wou mijn onderzoek toch over dit onderwerp doen. Gaandeweg heb ik wel de nodige informatie gevonden in tech-manuals van de IDEs, forums waar technische problemen besproken werden en boeken die de IDEs beschrijven.

Graag wil ik mijn promotor, Jan Willem, bedanken voor de begeleiding tijdens
deze bachelorproef. Daarnaast wil ik ook Leen Vuyge bedanken om op korte termijn mijn co-promotor te willen worden.