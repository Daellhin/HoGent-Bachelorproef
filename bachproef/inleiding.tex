%%=============================================================================
%% Inleiding
%%=============================================================================

% De inleiding moet de lezer net genoeg informatie verschaffen om het onderwerp te begrijpen en in te zien waarom de onderzoeksvraag de moeite waard is om te onderzoeken. In de inleiding ga je literatuurverwijzingen beperken, zodat de tekst vlot leesbaar blijft. Je kan de inleiding verder onderverdelen in secties als dit de tekst verduidelijkt. Zaken die aan bod kunnen komen in de inleiding~\autocite{Pollefliet2011}:

%\begin{itemize}
%    \item context, achtergrond
%    \item afbakenen van het onderwerp
%    \item verantwoording van het onderwerp, methodologie
%    \item probleemstelling
%    \item onderzoeksdoelstelling
%    \item onderzoeksvraag
%    \item \ldots
%\end{itemize}

\chapter{\IfLanguageName{dutch}{Inleiding}{Introduction}}
\label{ch:inleiding}

De job van een programmeur is complex met verschillende verantwoordelijkheden. Dit gaat van het effectieve schrijven van nieuwe code tot het testen van deze code, het opsporen van fouten, het beheren van verschillende versies... Om deze lasten te verminderen zijn er over de jaren heen verschillende geïntegreerde ontwikkelingsomgevingen (IDEs) op de markt gekomen, met het doel de productiviteit van de programmeur te verhogen. 

IDEs zijn ver gekomen sinds het idee ten eerste werd voorgesteld in de 1980 \autocite{Kline2005}. De eerste versies waren onhandig in gebruik en vereisten meerdere uren professionele training om effectief gebruikt te kunnen worden. Deze programma's hadden wel het potentieel om programmeurs productiever te laten worden, maar door de grote leercurve verkozen velen de gewone tekst-editor. De kost van de eerste software was ook een grote barrière, deze kon makkelijk oplopen tot \$20,000 per ontwikkelaar. Hedendaags gebruikt bijna iedere programmeur een IDE en deze zijn vaak gratis en open source.

Er is echter geen overzicht van de performantie van moderne IDEs. Om een geïnformeerde keuze te kunnen maken en niet enkel naar populariteit te kijken zou er een overzicht moeten bestaan. Dit onderzoek tracht dit te doen, om van enkele van de meest populaire IDEs de performantie te testen. De IDEs opgenomen in dit onderzoek zijn Visual Studio Code, Visual Studio, Eclipse, IntelliJ, en Notepad++. Deze komen voort uit de \textcite{StackOverflow2021} Developer Survey.

\newpage

\section{Onderzoeksvraag}

Dit onderzoek heeft als doel de volgende onderzoeksvragen te beantwoorden

\begin{itemize}
    \item Welke IDE is het meest performant op de aspecten van opstarttijd, zoektijd in bestanden, build tijd van code en CPU en RAM gebruik?
    \item Heeft de keuze van project en de programmeertaal waar die in geschreven is een effect op de opstart tijd, zoektijd in bestanden, build tijd van code en CPU en RAM gebruik?
\end{itemize}

\section{Onderzoeksdoelstelling}

In de vorm van een vergelijkend onderzoek worden eerst de voorgenoemde performantie aspecten getest met verschillende scripts en daarna vergeleken. Dit om uiteindelijk te kunnen bepalen wat de meest performante IDE op de markt is op algemene vlakken, maar ook extra toegelegd op Java en C\# programmatie.

\section{\IfLanguageName{dutch}{Opzet van deze bachelorproef}{Structure of this bachelor thesis}}
\label{sec:opzet-bachelorproef}

% Het is gebruikelijk aan het einde van de inleiding een overzicht te
% geven van de opbouw van de rest van de tekst. Deze sectie bevat al een aanzet
% die je kan aanvullen/aanpassen in functie van je eigen tekst.

De rest van deze bachelorproef is als volgt opgebouwd:

In Hoofdstuk~\ref{ch:stand-van-zaken} wordt een overzicht gegeven van de stand van zaken binnen het onderzoeksdomein, op basis van een literatuurstudie.

In Hoofdstuk~\ref{ch:methodologie} wordt de methodologie toegelicht en worden de gebruikte onderzoekstechnieken besproken om een antwoord te kunnen formuleren op de onderzoeksvragen.

Tenslotte wordt een antwoord geformuleerd op de onderzoeksvragen en de conclusie toegelicht op het onderzoek in Hoofdstuk~\ref{ch:conclusie}. Daarbij wordt er ook een aanzet gegeven voor toekomstig onderzoek als gevolg van een onverklaarde bevinding in deze studie.
