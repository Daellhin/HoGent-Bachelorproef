%%=============================================================================
%% Samenvatting
%%=============================================================================

% TODO: De "abstract" of samenvatting is een kernachtige (~ 1 blz. voor een
% thesis) synthese van het document.
%
% Deze aspecten moeten zeker aan bod komen:
% - Context: waarom is dit werk belangrijk?
% - Nood: waarom moest dit onderzocht worden?
% - Taak: wat heb je precies gedaan?
% - Object: wat staat in dit document geschreven?
% - Resultaat: wat was het resultaat?
% - Conclusie: wat is/zijn de belangrijkste conclusie(s)?
% - Perspectief: blijven er nog vragen open die in de toekomst nog kunnen
%    onderzocht worden? Wat is een mogelijk vervolg voor jouw onderzoek?
%
% LET OP! Een samenvatting is GEEN voorwoord!

%%---------- Nederlandse samenvatting -----------------------------------------
%
% TODO: Als je je bachelorproef in het Engels schrijft, moet je eerst een
% Nederlandse samenvatting invoegen. Haal daarvoor onderstaande code uit
% commentaar.
% Wie zijn bachelorproef in het Nederlands schrijft, kan dit negeren, de inhoud
% wordt niet in het document ingevoegd.

\IfLanguageName{english}{%
\selectlanguage{dutch}
\chapter*{Samenvatting}
\lipsum[1-4]
\selectlanguage{english}
}{}

%%---------- Samenvatting -----------------------------------------------------
% De samenvatting in de hoofdtaal van het document

\chapter*{\IfLanguageName{dutch}{Samenvatting}{Abstract}}

Vandaag de dag zijn er verschillende mogelijkheden voor software ontwikkelaars om geholpen te worden bij het schrijven van programmas door Integrated Developper Environements of IDEs, maar welke hiervan zijn het meest performant? Dit word onderzocht omdat hoe minder een ontwikkelaar op zijn IDE moet wachten hoe meer hij productief kan werken.

Er bestaan verschillende soorten IDEs op de markt. Sommige zijn gratis en sommige betalend. Sommige zijn toegelegd op een specifieke taal en anderen zijn algemeen bruikbaar en makkelijk customiseerbaar. Als laatste moet er ook het onderscheid gemaakt worden tussen echte IDEs, die het hele ontwikkelingsproces van een programma ondersteunen, en code editors die minimale functionaliteit hebben. De vraag die dit onderzoek wil beantwoorden is welke IDE het meest performant is op verschillende aspecten. Deze aspecten zijn de opstart tijd, de zoektijd in bestanden, de build tijd van code en het CPU en RAM gebruik in de idle stand. 
Ter beantwoording van de probleemstelling zijn volgende deelvragen geformuleerd

\begin{itemize}
    \item Welke IDE is het meest performant op de aspecten van opstart tijd, zoektijd in bestanden, build tijd van code en CPU en RAM gebruik?
    \item Heeft de keuze van project en de programmeertaal waar die in geschreven is een effect op de opstart tijd, zoektijd in bestanden, build tijd van code en CPU en RAM gebruik?
\end{itemize}

Het onderzoek werd uitgevoerd met 2 testprojecten: Cake en Mindustry, daarnaast werden er 5 IDEs onderzocht: VS Code, Visual Studio, Eclipse, Notepad++ en IntelliJ.

De verwachtingen waren dat VS Code het beste zou scoren op performantie aangezien dit de meest populaire en gebruikte IDE is. Dit werd ook bevestigd in dit onderzoek. Ook zijn Visual Studio en IntelliJ goede opties die meer functionaliteit aanbieden en een grotere performantie impact hebben.

Er werd ook opgemerkt dat de prestaties van de IDE grotendeels onafhankelijk zijn van de taal waarin het project geschreven.

Tot slot is er in dit onderzoek opgemerkt dat er bij het builden van C\# code een groot verschil in uitvoertijd, tot 2 maal zo lang, zit tussen de uitvoerende IDEs. Hier is meer onderzoek voor nodig aangezien deze dezelfde onderliggende commando's gebruiken.

De details van het onderzoek zijn te vinden in het volgende deel van deze scriptie.
