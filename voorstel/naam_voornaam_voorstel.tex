%==============================================================================
% Sjabloon onderzoeksvoorstel bachelorproef
%==============================================================================
% Gebaseerd op LaTeX-sjabloon ‘Stylish Article’ (zie voorstel.cls)
% Auteur: Jens Buysse, Bert Van Vreckem
%
% Compileren in TeXstudio:
%
% - Zorg dat Biber de bibliografie compileert (en niet Biblatex)
%   Options > Configure > Build > Default Bibliography Tool: "txs:///biber"
% - F5 om te compileren en het resultaat te bekijken.
% - Als de bibliografie niet zichtbaar is, probeer dan F5 - F8 - F5
%   Met F8 compileer je de bibliografie apart.
%
% Als je JabRef gebruikt voor het bijhouden van de bibliografie, zorg dan
% dat je in ``biblatex''-modus opslaat: File > Switch to BibLaTeX mode.

\documentclass{voorstel}

\usepackage{lipsum}
\usepackage{lmodern}

%------------------------------------------------------------------------------
% Metadata over het voorstel
%------------------------------------------------------------------------------

%---------- Titel & auteur ----------------------------------------------------

% TODO: geef werktitel van je eigen voorstel op
\PaperTitle{Het verhogen van productiviteit en tevredenheid door het kiezen van de juiste geïntegreerde ontwikkelingsomgeving(IDE)}
\PaperType{Onderzoeksvoorstel Bachelorproef 2021-2022} % Type document

% TODO: vul je eigen naam in als auteur, geef ook je emailadres mee!
\Authors{Lorin Speybrouck\textsuperscript{1}} % Authors
%\CoPromotor{Piet Pieters\textsuperscript{2} (Bedrijfsnaam)}
\affiliation{\textbf{Contact:}
  \textsuperscript{1} \href{mailto:lorin.speybrouck@student.hogent.be}{lorin.speybrouck@student.hogent.be};
  %\textsuperscript{2} \href{mailto:piet.pieters@acme.be}{piet.pieters@acme.be};
}

%---------- Abstract ----------------------------------------------------------

\Abstract{
    De job van een programmeur is complex en een verhoging in de productiviteit van ontwikkelaars kan leiden tot een reductie in ontwikkelingstijd voor nieuwe en bestaande projecten. Geïntegreerde ontwikkelingsomgevingen (IDEs) kunnen hierbij helpen, maar er is een groot aanbod van verschillende IDEs met verschillende features en prijzen. In deze bachelorproef wordt er onderzocht wat developers zelf waarderen en vereisen in hun ontwikkelingssoftware. Er wordt ook extra nadruk gelegd op het onderzoeken van de hardware requirements van verschillende IDEs. Dit zal gebeuren met een vragenlijst voor beginnende en ervaren programmeurs en technische testen voor specifieke IDEs. Dit onderzoek zou kunnen helpen bij het kiezen van de geschikte ontwikkelingsomgeving(en), zij deze betalend of gratis.
}

%---------- Onderzoeksdomein en sleutelwoorden --------------------------------
% TODO: Sleutelwoorden:
%
% Het eerste sleutelwoord beschrijft het onderzoeksdomein. Je kan kiezen uit
% deze lijst:
%
% - Mobiele applicatieontwikkeling
% - Webapplicatieontwikkeling
% - Applicatieontwikkeling (andere)
% - Systeembeheer
% - Netwerkbeheer
% - Mainframe
% - E-business
% - Databanken en big data
% - Machineleertechnieken en kunstmatige intelligentie
% - Andere (specifieer)
%
% De andere sleutelwoorden zijn vrij te kiezen

\Keywords{Applicatieontwikkeling. Geïntegreerde ontwikkelingsomgevingen (IDEs) --- Hardware vereisten --- Gebruikersgerichte software} % Keywords
\newcommand{\keywordname}{Sleutelwoorden} % Defines the keywords heading name

%---------- Titel, inhoud -----------------------------------------------------

\begin{document}

\flushbottom % Makes all text pages the same height
\maketitle % Print the title and abstract box
\tableofcontents % Print the contents section
\thispagestyle{empty} % Removes page numbering from the first page

%------------------------------------------------------------------------------
% Hoofdtekst
%------------------------------------------------------------------------------

% De hoofdtekst van het voorstel zit in een apart bestand, zodat het makkelijk
% kan opgenomen worden in de bijlagen van de bachelorproef zelf.
%---------- Inleiding ---------------------------------------------------------

\section{Introductie} % The \section*{} command stops section numbering
\label{sec:introductie}

De job van een programmeur is complex met verschillende verantwoordelijkheden. Dit gaat van het effectieve schrijven van nieuwe code tot het testen van deze code, het opsporen van fouten, het beheren van verschillende versies... Om deze lasten te verminderen zijn er over de jaren heen verschillende geïntegreerde ontwikkelingsomgeving (IDEs) op de markt gekomen, met het doel de productiviteit van de programmeur te verhogen. 

IDEs zijn ver gekomen sinds het idee ten eerste werd voorgesteld in de 1980 \autocite{Kline2005}. De eerste versies waren onhandig in gebruik en vereisten meerdere uren professionele training om effectief gebruikt te kunnen worden. Deze programma’s hadden wel het potentieel om programmeurs productiever te laten worden, maar door de grote leercurve verkozen vele de gewone tekst-editor. De kost van de eerste software was ook een grote barrière, deze kon makkelijk oplopen tot \$20,000 per ontwikkelaar. Hedendaags gebruikt bijna iedere programmeur een IDE en deze zijn vaak gratis en open source. 

In dit onderzoek wordt getracht te achterhalen hoe het met de markt van gratis IDEs gesteld is, op vlakken van hardware vereisten, leercurve, functionaliteit, configuratie en extensies. Daarnaast word onderzocht in welke use cases beginnende en ervaren programmeurs de betalende software beter ervaren.

%---------- Stand van zaken ---------------------------------------------------

\section{State-of-the-art}
\label{sec:state-of-the-art}

De hoofdfunctie van de geïntegreerde ontwikkelingsomgeving is de broncode-editor, deze is meestal voorzien van syntax markering, auto aanvulling en code controle. Daarnaast heeft de moderne IDE een ingebouwde debugger en versie controlesysteem. Momenteel is de populairste IDE Visual Studio Code \autocite{StackOverflow2021}, deze gratis software is vooral populair door zijn kleine leercurve en ecosysteem van extensies. Een andere populaire gratis IDE is Eclipse, deze is gespecialiseerd in Java, maar heeft vele extensie voor andere programmeertalen. Visual Studio 2019 is een betalende IDE, met een gratis versie, en heeft ingebouwde ondersteuning voor C, C++, C\#, F\#, TypeScript, XML en HTML. Een andere populaire optie zijn de IDEs van JetBrains, dit bedrijf biedt meerder betalende IDEs aan met bijvoorbeeld IntelliJ voor Java-applicaties en Rider voor C++ applicaties.

In het verleden zijn er al gebruiksanalyses van Eclipse en Visual Studio geweest \autocite{Murphy2006, Amann2016}. Hier werd er passief opgenomen welke functionaliteiten van de IDE de programmeurs meest gebruikten, maar er is nog geen onderzoek gebeurd naar wat programmeurs zelf waarderen in IDEs. 

Er is geen bestaand onderzoek over de hardware requirements van IDEs en hoe de verschillende programma’s hier zich van elkaar onderscheiden. Het is wel algemeen aanvaard dat voor sommige projecten de hardware requirements zeer hoog worden. Over dit onderwerp is er al onderzoek gebeurd voor het migreren van desktop gebaseerde IDEs naar de cloud \autocite{Devadiga2021}.

% Voor literatuurverwijzingen zijn er twee belangrijke commando's:
% \autocite{KEY} => (Auteur, jaartal) Gebruik dit als de naam van de auteur geen onderdeel is van de zin.
% \textcite{KEY} => Auteur (jaartal)  Gebruik dit als de auteursnaam wel een functie heeft in de zin (bv. ``Uit onderzoek door Doll & Hill (1954) bleek  ...'')

%---------- Methodologie ------------------------------------------------------
\section{Methodologie}
\label{sec:methodologie}

Dit onderzoek zal bestaan uit zowel een vragenlijst als technische experimenten. De vragenlijst die zal opgesteld worden zal ingevuld worden door zowel beginnende als ervaren programmeurs. Deze vragenlijst zal nagaan welke eigenschappen elke programmeur belangrijk acht, en welke functionaliteiten men dagelijks gebruikt. Daarnaast zal er ook gevraagd worden met welke IDEs men ervaring heeft en welke men verkiest. Ten laatste zal er nagegaan worden welke problemen men ondervindt in het dagelijkse gebruik van IDEs en wat men mist van functionaliteit.

De technische experimenten zullen bestaan in het meten van opstart en compileer -tijden in zowel run als debug configuratie, en het systeemverbruik bij normaal gebruik en op de achtergrond. Hiervoor zullen 3 projecten gebruikt worden: een leeg project met geen code, een project met een applicatie van gemiddelde grote, en een zware applicatie met meerdere dependencies. De IDEs die opgenomen zullen worden in het onderzoek zijn: Visual Studio Code, Eclipse, Visual Studio 2019 Community, Visual Studio 2019 Enterprise met en zonder dotUltimate extensies, Rider en IntelliJ.



%---------- Verwachte resultaten ----------------------------------------------
\section{Verwachte resultaten}
\label{sec:verwachte_resultaten}

De auteur verwacht dat beginnende gebruikers vooral de auto aanvulling en code controle belangrijk zullen vinden. De auteur verwacht ook dat de ervaren gebruikers dezelfde functionaliteiten belangrijk zullen vinden met daarnaast de geavanceerdere features die voornamelijk helpen bij het beheren van een grote code base die vooral voorkomen in de gespecialiseerde betalende IDEs.

Voor de technische experimenten verwacht de auteur dat Visual Studio Code het beste zal scoren op opstart en compileer -tijden maar dat het resource verbruik bij normaal gebruik en op de achtergrond het grootste zal zijn. De andere IDEs zullen meer tijd in beslag nemen om op te starten maar de compileer tijden zullen gelijkaardig zijn.



%---------- Verwachte conclusies ----------------------------------------------
\section{Verwachte conclusies}
\label{sec:verwachte_conclusies}

De verwachte resultaten leiden er toe dat Visual Studio Code de aanbevolen IDE is voor de gemiddelde programmeur, zoals de huidige markt weerspiegeld. Voor sommige gebruikers kan het wel voordelig zijn om gespecialiseerde IDEs aan te schaffen, maar enkel als hier een specifieke nood toe is.



%------------------------------------------------------------------------------
% Referentielijst
%------------------------------------------------------------------------------
% TODO: de gerefereerde werken moeten in BibTeX-bestand ``voorstel.bib''
% voorkomen. Gebruik JabRef om je bibliografie bij te houden en vergeet niet
% om compatibiliteit met Biber/BibLaTeX aan te zetten (File > Switch to
% BibLaTeX mode)

\phantomsection
\printbibliography[heading=bibintoc]

\end{document}
